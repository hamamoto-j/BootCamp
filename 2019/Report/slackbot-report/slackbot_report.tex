\documentclass[12pt]{jsarticle}
\usepackage[dvipdfmx]{graphicx}
\textheight = 25truecm
\textwidth = 18truecm
\topmargin = -1.5truecm
\oddsidemargin = -1truecm
\evensidemargin = -1truecm
\marginparwidth = -1truecm

\def\theenumii{\Alph{enumii}}
\def\theenumiii{\alph{enumiii}}
\def\labelenumi{(\theenumi)}
\def\labelenumiii{(\theenumiii)}
\def\theenumiv{\roman{enumiv}}
\def\labelenumiv{(\theenumiv)}
\usepackage{comment}
\usepackage{url}


%%%%%%%%%%%%%%%%%%%%%%%%%%%%%%%%%%%%%%%%%%%%%%%%%%%%%%%%%%%%%%%%
%% sty/ にある研究室独自のスタイルファイル
\usepackage{jtygm}  % フォントに関する余計な警告を消す
\usepackage{nutils} % insertfigure, figref, tabref マクロ

\def\figdir{./figs} % 図のディレクトリ
\def\figext{pdf}    % 図のファイルの拡張子

\begin{document}
%%%%%%%%%%%%%%%%%%%%%%%%%%%%
%% 表題
%%%%%%%%%%%%%%%%%%%%%%%%%%%%
\begin{center}
{\LARGE SlackBotプログラムの報告書}
\end{center}

\begin{flushright}
  2019/4/26\\
  浜本 時空
\end{flushright}
%%%%%%%%%%%%%%%%%%%%%%%%%%%%
%% 概要
%%%%%%%%%%%%%%%%%%%%%%%%%%%%
\section{概要}
\label{sec:introduction}
本資料は2019年度新人研修課題にて作成したSlackBotプログラムの報告書である.本資料では,課題内容,理解できなかった部分,作成できなかった機能,および自主的に作成した機能について述べる.
なお,本資料において発言とはチャットツールであるSlack\cite{slack}の特定のチャンネル上で発言すること,または発言そのものを指す.また,本資料において発言内容は``''で囲って表す.

\section{課題の内容}
\begin{enumerate}
  \item 任意の文字列を発言するプログラムの作成\\
  受信した発言の中に``「任意の文字列」と言って''という文字列を含む場合は,``任意の文字列''と発言するプログラムを作成する
  \item SlackBotプログラムへの機能追加\\
  SlackBotプログラムへ機能を追加する. Slack以外のWebサービスのAPIやWebhookを利用した機能を追加する.
\end{enumerate}

\section{理解できなかった部分}
理解できなかった部分を以下に示す.
\begin{enumerate}
  \item 今回の課題ではサンプルプログラムが配布されており,サンプルプログラム内に\verb|naive_respond|メソッドが存在する.このメソッドを使用するとSlackにメッセージが表示されるが,メソッド内にIncoming WebhookのURLの記述がないにもかかわらず,なぜSlack上にメッセージが表示されるのかが理解できなかった
\end{enumerate}

\section{作成できなかった機能}
作成できなかった機能を以下に示す.
\begin{enumerate}
  \item 設定したOutgoing Webhook以外からのPOSTの拒否
  \item 表示させる商品の値段の閾値を決める機能
\end{enumerate}

\section{自主的に作成した機能}
自主的に作成した機能を以下に示す.
\begin{enumerate}
  \item 機能としてサポートされていない発言を受信した場合は,機能の概要を発言する.詳しくはSlackBotプログラムの仕様書に記載する.
\end{enumerate}

\bibliographystyle{ipsjunsrt}
\bibliography{mybibdata}

\end{document}
